\section{Предлагаемое решение 1}

\begin{frame}
\frametitle{\insertsection} 
\framesubtitle{\insertsubsection}
Игра, многие параметры которой могут быть настроены для занятий с пациентами.
    \begin{itemize}
        \item Скорость полёта шарика и ход игрового времени могут быть настроены.
        
        \item Когда мячик подлетает к половине стола игрока, игровое время может замедляться, чтобы дать возможность игроку сделать движение, и классификатору --- чтобы его распознать.
        
        \item  В простейшем варианте игра может сама "играть" в себя. Человеку нужно будет только сделать какое-то движение (например, сжать кулак), чтобы бот, который играет за человека, сам отбил шарик.
        
        \item Игровой экран может быть поделён на зоны, поверхность стола --- также, чтобы вводить новые ситуации в игру, на которые нужно будет отвечать новыми движениями.
    \end{itemize}
\end{frame}

\section{Предлагаемое решение 2}

\begin{frame}
\frametitle{\insertsection} 
\framesubtitle{\insertsubsection}
    \begin{itemize}
        \item В реальности техника игры очень комплексна (двигаются руки, кисти, даже пальцы участвуют в игре), что даёт большой простор на введение новых движений.
        
        \item Игрок будет видеть, как ракетка на экране отвечает его движениям, что важно для способствования восстановлению двигательных функций.
        
        \item Возможность сетевой игры вводит элемент конкуренции, что способствует большей целеустремлённости к занятиям и выздоровлению.
    \end{itemize}
\end{frame}

